\documentclass{../beamer_template/myBeamer}

\input{../beamer_template/header.tex}

\title[Case study model]{Case study: an epidemiological model}
%\subtitle{Course and practical application}
\author[Short Author]{Author}
\date{\today}
\institute{Institut\\
\bigskip
\includegraphics[scale=2]{../beamer_template/figures/logos/openmole.png}}

\begin{document}



\begin{frame}[plain]
	\titlepage
\end{frame}
\addtocounter{framenumber}{-1}

\AtBeginSection[]
{
	\frame{
		\tableofcontents[currentsection, hideallsubsections]
	}
	\addtocounter{framenumber}{-1}
}


% no need for outline
%\sframe{Outline}{
%\tableofcontents
%}

%Introducing a model 

%general purpose : spatial epidemio model

%scenarization

% - pedestrian dynamics



\sframe{Decision making in a chaotic world}{

 % contextualize Center for Zombie Research
 % knowledge forgotten about how to use simulation models ? -> you are the last hope of the world

}

\sframe{An operational model}{

}


\sframe{Overview of the model}{

}

\sframe{Basic processes and parameters}{

}


\sframe{Pedestrian simulation}{

  % a bit of literature on pedestrian models

}


\sframe{Information and rescues}{

}

\sframe{A flexible and more general model}{

% one world on multi-modeling / deactivated parameters
% hidden parameters ?

}


\sframe{The model in practice}{
 % scala impl + GUI -> layus on implementation / platform

}


\sframe{Let's get your hands on it}{
  
  \begin{itemize}
  	\item Try the GUI and changing parameters
  	\item Most of next courses will be based on that model (additional processes will be detailed when needed)
  \end{itemize}
  
}



\backupbegin




%%%%%%%%%%%%%%%%%%%%%
\begin{frame}[allowframebreaks]
\frametitle{References}
\bibliographystyle{apalike}
\bibliography{biblio}
\end{frame}
%%%%%%%%%%%%%%%%%%%%%%%%%%%%


\backupend





\end{document}


