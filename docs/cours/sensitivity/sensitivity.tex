\documentclass{../beamer_template/myBeamer}

\input{header.tex}

\title[DOE/Sensitivity Analysis]{Design of Experiments and Sensitivity analysis}
\subtitle{Course and practical application}
\author[Short Author]{Author}
\date{\today}
\institute{Institut\\
\bigskip
\includegraphics[scale=2]{../beamer_template/figures/logos/openmole.png}}

\begin{document}



\begin{frame}[plain]
	\titlepage
\end{frame}
\addtocounter{framenumber}{-1}

\AtBeginSection[]
{
	\frame{
		\tableofcontents[currentsection, hideallsubsections]
	}
	\addtocounter{framenumber}{-1}
}


\sframe{Outline}{
\tableofcontents
}


\section{Introduction}


\sframe{Design of Experiments}{

 - Interactive model exploration by hand and the need for preliminary experiments
 - The Design of Experiments as the definition of tasks to extract information from the simulation model
 - Example: NetLogo behavior space: basic grid DOE
 - Sensitivity analysis as an advanced DOE

*Remark 1: terminology strongly depends on disciplines and practices*

*Remark 2: these are generally **preliminary experiments** to prepare more elaborated, question-related, experiments*


}



\section{Basic experiments}


\sframe{Basic experiments}{

*Provide explicitly sampling points on which the model (or its replication task) will be run: notion of **direct sampling** in OpenMOLE (corresponds to DOE in the literature)*

 - full samplings
 - elaborated sampling for high dimensions given a low computational budget (**the curse of dimensionality**)

}


\sframe{One factor at a time}{
  
  

All factors have nominal values and a discrete variation set, in which each is varied while others remaining fixed

 - *when model is slow - or computational budget highly limited* 
 - *does not capture interaction between parameters, and highly dependent on nominal values*
 - *seen as a bad practice* BUT *useful for models taking significant time, and prone to thematic interpretation*


%<img width=600 src=https://miniocodimd.openmole.org:443/codimd/uploads/upload_82c20b2e74a3f5b96a0fc227311cd516.png/>

*Example where One-At-a-Time fails*
  
}


\sframe{Grid sampling}{



Ensemble product of discrete variation ranges for factors (usually a regular grid but not necessarily)

*quickly limited by the curse of dimensionality - in practice still powerful with a quick model and a low number of parameters*

*naive approach, i.e. done by many "simulation-newcomers" such as economics or some parts of physics*

}


\section{DOE Samplings}


\sframe{DOE Samplings}{


*Computational limitations => need specific methods to efficiently sample the parameter space*

The field of Design of Experiments has proposed different methods for numerical experiments given limited computational resources

Examples: Sobol sequence (quicker convergence of integral estimation), Latin Hypercube Sampling, Orthogonal sampling

}


\sframe{Latin Hypercube Sampling}{

*Minimizing discrepency: intuitively being spread evenly accross the parameter space*
(def of discrepancy)

|x|||||
|:--:|:--:|:--:|:--:|:--:|
||x||||
|||||x|
||||x||
|||x|||

*Latin cube: one point in each row and column; hypercube generalization in any dimension*

}



\sframe{Sobol sequence}{

*Quasi-random sequences with low discrepancy (also Halton sequences e.g.)*

Estimate integral in $1/N$ instead of $1/\sqrt{N}$ with random sampling

Constructed recursively (using bit representations).

TODO illustration in 2D

}



\section{Sensitivity analysis}


\sframe{Sensitivity analysis}{

*How to summarize model sensitivity and isolate principal factors ?*

Examples: Morris and Saltelli methods

}


\sframe{Morris method}{


*Idea :* Sample trajectories in the parameter space in a One-At-a-Time manner. Screening method isolating *elementary effects*

\cite{saltelli2004sensitivity}

 - isolate local effects of factors
 - more efficient than point sampling to get individual effects
 - useful as a first experiment to understand the relative influence of factors

\cite{campolongo2011screening} propose to extend the method with Sobol sequences




}


\sframe{Saltelli method}{

Estimation of relative and conditional variances

\[
ST_i = \frac{E_{\mathbf{X}\sim i}\left[Var(Y | \mathbf{X}\sim i) \right]}{Var(Y)}
\]

}



\section{Application in OpenMOLE}


\sframe{OpenMOLE syntax: Direct sampling}{


%\begin{lstlisting}
val explo = DirectSampling(
    evaluation = model,
    sampling = ...
)
%\end{lstlisting}

}


\sframe{Example of samplings}{

 - One-factor sampling
```
  sampling = OneFactorSampling(
    (x1 in (0.0 to 1.0 by 0.2)) nominal 0.5,
    (x2 in (0.0 to 1.0 by 0.2)) nominal 0.5
  )
```

 - Grid sampling
```
  sampling = (x1 in (0.0 to 1.0 by 0.5)) x (x2 in (0.0 to 1.0 by 0.5))
```

 - LHS Sampling

```
  sampling = LHS(100,x1 in (0.0,1.0),x2 in (0.0,1.0))
```

 - Sobol sampling

```
  sampling = SobolSampling(100,x1 in (0.0,1.0),x2 in (0.0,1.0))
```

}


\sframe{Saltelli}{

(method in itself)

```
val sen = SensitivitySaltelli(
  //evaluation = (model on env),
  evaluation = (model on env by 1000),
  samples = 100000,
  inputs = Seq(humanFollowProbability in (0.0,1.0), humanInformedRatio in (0.0,1.0),humanInformProbability in (0.0,1.0)),
  outputs = Seq(peakTime, peakSize, totalZombified,halfZombified, spatialMoranZombified,spatialDistanceMeanZombified,spatialEntropyZombified,spatialSlopeZombified),
  )
```

}



\sframe{Morris}{

(example from market)
```
SensitivityMorris(
    evaluation = modelExec on envLocal hook storeSimuCSV,
    inputs = Seq(inputNumberOfCars in (1.0, 41.0), 
                inputAcceleration in (0.0, 0.0099),
                inputDeceleration in (0.0, 0.099)
                ),
    outputs = Seq(outputSpeedMin, outputSpeedMax),
    repetitions = 100,
    levels = 5)
```
}



\section{Practical application}


\sframe{Practical application}{


Your turn to run some direct samplings and/or sensitivity analysis

\begin{itemize}
	\item given the described zombie model, what first experiment beyond stochasticity would be relevant ?
    \item write a script
    \item explore results (using e.g. the OpenMOLE GUI plots)
\end{itemize}

\textit{Resources:}
 - one script running directsampling
 - example of grid explo results
 - example of Saltelli

}






\backupbegin

\appendix


\sframe{Results of grid exploration}{

\textit{Cooperation model}

}




%%%%%%%%%%%%%%%%%%%%%
\begin{frame}[allowframebreaks]
\frametitle{References}
\bibliographystyle{apalike}
\bibliography{biblio}
\end{frame}
%%%%%%%%%%%%%%%%%%%%%%%%%%%%


\backupend





\end{document}


