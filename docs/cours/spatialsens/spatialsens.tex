\documentclass{../beamer_template/myBeamer}

\input{../beamer_template/header.tex}

\title[Spatial sensitivity]{Spatial sensitivity analysis}
\subtitle{Course and practical application}
\author[Short Author]{Author}
\date{\today}
\institute{Institut\\
\bigskip
\includegraphics[scale=2]{../beamer_template/figures/logos/openmole.png}}

\begin{document}



\begin{frame}[plain]
	\titlepage
\end{frame}
\addtocounter{framenumber}{-1}

\AtBeginSection[]
{
	\frame{
		\tableofcontents[currentsection, hideallsubsections]
	}
	\addtocounter{framenumber}{-1}
}


\sframe{Outline}{
\tableofcontents
}



\section{Introduction}

\sframe{Context}{

*Classical problems in geography / spatial sciences : MAUP, scale dependency, spatial non-stationarity*

}



\sframe{New approach by OpenMOLE}{

=> spatial configuration are parameters too

 - Space matters
 - Synthetic generators
 - Sensitivity to data noise
 
}



\sframe{Contents of this module}{

 %- Generation of spatial synthetic data
 %    - microscopic scale (buildings)
 %    - mesoscopic scale (population grid)
 %    - macroscopic scale (system of cities)
 %- Perturbation of real datasets
 %- Spatial indicators for model outputs
 %- Practice

	\tableofcontents

}



\section{Spatial synthetic data}



\sframe{General context}{

}



\sframe{Generating building layouts}{

\textit{At the microscopic scale (district): building layouts}

\cite{raimbault2019generating}

}



\sframe{}{

%<img width=300 src=https://miniocodimd.openmole.org:443/codimd/uploads/upload_260338e6eab71fba9ae929e390722ff9.png/>
%<img width=350 src=https://miniocodimd.openmole.org:443/codimd/uploads/upload_0b1e4cd937119a5cbbed947ba658f31b.png/>

}

\sframe{Results}{

%![](https://miniocodimd.openmole.org:443/codimd/uploads/upload_58272921ff9d3cbd03ec12560b63c5a5.png)

}


\sframe{Point cloud}{

%![](https://miniocodimd.openmole.org:443/codimd/uploads/upload_f3622e53ef9ccf5b7fad1eeff3929eae.png)

}





\sframe{Population grid}{

*At the mesoscopic scale: population grid*

\cite{raimbault2018calibration}

 - Reaction-diffusion model
 - Urban form measures

}


\sframe{}{

%![](https://miniocodimd.openmole.org:443/codimd/uploads/upload_f2f2a929afa3162e5a2f5df90227a692.png)

}



\sframe{PSE on the morphological space}{

%<img width=400 src=https://miniocodimd.openmole.org:443/codimd/uploads/upload_cc5a01cb6a529febbde6107979e6ee32.png/>

}




\sframe{Synthetic systems of cities}{


*At the macroscopic scale: systems of cities*

 - Evolutive urban theory: systems of cities follow general stylized facts 
 - rank-size law
 - central place theory

}


\section{Perturbation of data}


\sframe{Real data perturbation}{

 *How does noise in real data impacts the result ?*

 - WIP

 *How does perturbation of real data allows to explore scenario*
 - Forcity example

}


\section{Spatial indicators for model outputs}


\sframe{Spatial statistics}{

*In the spatial approach, spatial model indicators are also important: what kind of spatial structure does the model produce ?*

 
 - previous form indicators at different scales
 - spatial statistics

}


\sframe{Spatial form as indicators}{


*spatial correlations ?*

}


\sframe{Spatial statistics}{

(examples)

}


\sframe{Moran index}{

*Spatial autocorrelation at a given range*

Given spatial weights $w_{ij}$

\[
I = \frac{N}{\sum_{i,j} w_{ij}} \cdot \frac{\sum_{i,j}w_{ij} \cdot (X_i - \bar{X}) (X_j = \bar{X})}{\sum_i (X_i - \bar{X})^2}
\]

}


\sframe{Optimal autocorrelation spatial scales}{

% example from EnergyPrice ?

}

\sframe{Ripley K function}{

*Quantifying level of clustering regarding a null model*

}


\sframe{Geographically Weighted Regression}{


}





\section{Application: sensitivity to spatial configuration}



\sframe{Method flowchart}{

%<img width=500 src=https://miniocodimd.openmole.org:443/codimd/uploads/upload_8a2d364caade89b648fd1ae36cff9362.png/>

}


\sframe{Quantification of spatial sensitivity}{

Relative distance of phase diagrams

$$ d_r\left(\mu_{\vec{\alpha}_1},\mu_{\vec{\alpha}_2}\right) = 2 \cdot \frac{d(\mu_{\vec{\alpha}_1},\mu_{\vec{\alpha}_2})^2}{Var\left[\mu_{\vec{\alpha}_1}\right] + Var\left[\mu_{\vec{\alpha}_2}\right]}
$$

}


\sframe{Application: Schelling model}{

*Why could the Schelling model be sensitive to space ?*

\cite{banos2012network}

%<img width=250 src=https://miniocodimd.openmole.org:443/codimd/uploads/upload_803863b93b3c8fa6b590c3c50c8d41ba.png/>
%<img width=250 src=https://miniocodimd.openmole.org:443/codimd/uploads/upload_7b0ec80e4f2f29275ce1d0e4a7521b1f.png/>

}


\sframe{Sensitivity of the Schelling model}{

%![](https://miniocodimd.openmole.org:443/codimd/uploads/upload_3e8af699ecfb51a74499b331972d3f7d.png)

}


\sframe{Application: Sugarscape model}{

*A model of resource collection*

}











\backupbegin


%%%%%%%%%%%%%%%%%%%%%
\begin{frame}[allowframebreaks]
\frametitle{References}
\bibliographystyle{apalike}
\bibliography{biblio}
\end{frame}
%%%%%%%%%%%%%%%%%%%%%%%%%%%%


\backupend





\end{document}


